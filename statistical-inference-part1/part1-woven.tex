---
title: "Part 1: Exponential Distributions and the Central Limit Theorem"
author: "Shayonendra Nath Tagore"
date: "25 May, 2020"
output:
  pdf_document:
    number_sections: true
    toc: false
#    toc_depth: 3
    fig_caption: yes
    includes:
      in_header: my_header.tex
    fig_width: 4.66
    fig_height: 3
header-includes:
  - \usepackage{booktabs}
  - \usepackage{siunitx}
---



# Introduction

This assignment will explore whether the Central Limit Theory
([CLT](https://en.wikipedia.org/wiki/Central_limit_theorem#Classical_CLT)) can be applied to
exponential distributions. The CLT states that the distribution of averages for Independent and
Identically Distributed (iid) variables will approximate a normal distribution as sample size
increases.

This will be explored through three steps:

1. Properties of a single exponential distribution
2. Distribution of Averages for 1000 Exponential Distributions
3. Comparison of Exponential and Normal Distributions


# Properties of a Single Exponential Distribution

Due to the random selection of values, each sample is only representative of the population. This
representation increases in accuracy as the number of values (\( n \)) increases, ultimately approaching \( \mu \). As such, we will
make n = 1000. Exponential Distributions are modeled around \(\lambda\), with the mean \( \mu
= \frac{1}{\lambda} \). Lets assume \(\lambda = 1/5 (0.2)\). This sample will looks as such:

<img src="figure/sample-1.png" title="Histogram representing the spread of a single exponential distribution sample with 1000 values." alt="Histogram representing the spread of a single exponential distribution sample with 1000 values." style="display: block; margin: auto;" />

A necessity for CLT distributions is that it follows the Law of Large Numbers. As \( n \) approaches
infinity, the sample mean will become asymptotic along \( \mu \). This asserts that the each sample
results follow a predictable pattern. This can be assessed by observing changes in the average as \(
n_i \) increases. The Expected Value (EV) for this sample is:

\[ E[N] = \frac{1}{\lambda} = \frac{1}{0.2} = 5 \]

This single distribution has \( E[N] = 5 \). This is confirmed through Fig.2 below. As \( N_i \)
increases, the average approaches 5.

<img src="figure/lln_plot-1.png" title="Change in sample mean as sample size increases in an exponential distribution with 1000 values. As $\lambda$ = 1/5, the population mean is 5. The blue line is the average of the 1000 values, and as the number of values increases, the sample mean approximates the population mean of 5." alt="Change in sample mean as sample size increases in an exponential distribution with 1000 values. As $\lambda$ = 1/5, the population mean is 5. The blue line is the average of the 1000 values, and as the number of values increases, the sample mean approximates the population mean of 5." style="display: block; margin: auto;" />

# Distribution of Averages for 1000 Simulations

To better understand how increasing \( n \) leads to approximation of the population, we will
examine a large number of samples with 40 values each.  Due to random selection, each sample alone
will not be an accurate representation of the population. If exponential distributions follow the
CLT, the mean of each sample will create a distribution of averages which in turn represent the
population.

This section will explore many exponential distributions using the following properties:
- \( \lambda \) = 0.2
- n = 40 per sample
- 1000 samples

Plotting the distribution of averages gives this:

<img src="figure/large_scale_exploration_largescale-1.png" title="Distribution of averages for 1000 exponential distributions. The sample SD and SE are almost overlapping." alt="Distribution of averages for 1000 exponential distributions. The sample SD and SE are almost overlapping." style="display: block; margin: auto;" />

Plotting the distribution of averages gives the figure above. Visually, it is not hard to imagine
the shape of normal curve. It is easier than imaging the bell curve on the single distribution
represented in Figure 5, an advantage given through use of distribution of averages. However,
further statistical analysis is required to confirm that distribution of averages simulates a normal
distribution. As with the single expontential distribution above, the \( E[N] = 5 \) as \( \lambda =
0.2 \). While the average of the distributions do not equal 5 but 5.0353567, differences can be
explained through Standard Error (SE). SE is calculated using:

\[ SE = \frac{\sigma}{\sqrt{n}} \approx 0.79 \]

As such, the EV of 5 falls in the range of \( 5.04 \pm 0.79 \). The SD of the distribution of averages is 0.8004739 (population SD is 5, using \( \sigma = \sqrt{\frac{1}{\lambda^2}} \)). The sample SD appears to be a strong predictor for the population SD. Further confirmation can be found through hypothesis testing.



Since the distribution of means approximates the population mean, the last step is to check if
exponential distributions follows the CLT and becomes normally distributed. Figure 4 suggests that
the distribution of averages follows the normal distribution. This is done by assessing whether the
SD of the distribution of averages approximates the SD of a normal distribution. Using the null
hypothesis that distribution of averages represents a normal distribution, with mean \( \mu \) and
standard deviation \( \sigma \) and hypothesis that it does not represent a normal distribution. The
calculated P-value is 0.96. With an \( \alpha = 0.05 \), we fail to reject
the null. Thus, we conclude that exponential distributions follow the CLT.

\(\pagebreak\)

# Appendix


```r
### loading all packages
pacman::p_load(knitr,
               dplyr,
               ggplot2,
               ggthemr)
ggthemr::ggthemr("fresh")

# for greater stability while exploring where no one
# (except my peers) has gone before
set.seed(1702)

### generating a single exponential distribution
sample_num <- 1000
sample <- rexp(sample_num, rate = 0.2)

### plotting a single exponential distribution
sample %>% data.frame(num = .) %>%
    ggplot(aes(x = num)) +
    geom_histogram(bins = 30, col = ggthemr::swatch()[8]) +
    xlab("Value") + ylab("Count") + ggtitle("Exponential Distribution")
means <- cumsum(sample)/(1:sample_num)

### plotting to see if exponential distributions follow the law of large numbers
means <- cumsum(sample)/(1:sample_num)
asymp_point <- round(mean(means), digits=1)
means %>%
    data.frame(x=1:sample_num, y = .) %>%
    ggplot(aes(x=x, y=y)) +
    geom_point() +
    geom_hline(yintercept = mean(means)) +
    xlab("Value") + ylab("Average") + ggtitle("Sample Mean Over Sample Size")

### Plotting the distribution of averages for 1000 exponential distributions.
sample_averaging_num  <- 40
sim_num <- 1000
mns = NULL
for (i in 1:sim_num) mns = c(mns, mean(rexp(sample_averaging_num, rate = 0.2)))
mns_SE <- (1/0.2)/sqrt(sample_averaging_num) # the Standard Error
mns %>%  data.frame(num = .)  %>%
ggplot(aes(x = num)) +
    geom_histogram(col = ggthemr::swatch()[8], bins = 30) +
    geom_vline(aes(xintercept = mean(mns), color = "mean")) +
    geom_vline(aes(xintercept = 5, color = "EV")) +
    geom_vline(aes(xintercept = 5+mns_SE, color = "SE")) +
    geom_vline(aes(xintercept = 5-mns_SE, color = "SE")) +
    geom_vline(aes(xintercept = sd(mns)+5, color = "Sample.SD")) +
    geom_vline(aes(xintercept = -sd(mns)+5, color = "Sample.SD")) +
    scale_color_manual(name = "statistics", values = c(mean = ggthemr::swatch()[4],
                                                       EV = ggthemr::swatch()[5],
                                                       SE = ggthemr::swatch()[6],
                                                       Sample.SD = ggthemr::swatch()[2])) +
    theme(legend.position = "right")

### Hypothesis testing on the distribution of averages for exponential distribution
z_stats <- abs((mean(mns)-5)/mns_SE)
p_val <- 2*pnorm(z_stats, lower.tail = FALSE)
```
